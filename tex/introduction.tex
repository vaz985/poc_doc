% Modelo americo/mario de intro
Proteger informação sensível do público é um objetivo importante de segurança. O campo do Fluxo Quantitativo de Informação(QIF) se preocupa em quantificar quanto de informação sensível um sistema vaza, e tem sido muito ativo na última década [1]-[X]. 

A representação do sistema é chamada de \emph{Canal} e é a distribuição de probabilidade das saídas de cada entrada, essa definição modela o comportamento do sistema.

Em sistemas complexos derivar o canal diretamente não é trivial, ao contrario de sistemas pequenos/simples.
%O problema com a modelagem é que, intuitivamente, a abordagem inicial é pensar no sistema como uma coisa só, que é suficiente para sistemas simples mas para sistemas robustos não é trivial.
A partir disso, foi proposto uma abordagem aproximativa que modela o canal da composição de partes do sistema e é proposto operadores que capturam as interações que ocorrem entre os componentes. %Falar sobre a abordagem tradicional
Essa abordagem simplifica o processo de modelagem. % Faz mais alguma coisa?

Com a orientação do professor Mário S. Alvim, o objetivo desse projeto é comparar a análise de fluxo de informação pela aproximação usando os operadores e o calculo exato no canal, e modelar novos protocolos como composição de canais e analisar sua vunerabilidade.
Na primeira parte do Projeto Orientado em Computação será feita uma comparação do vazamento de informação em protocolos já estudados, usando as diferentes abordagens, e será implementada uma biblioteca em C++ para auxiliar esse projeto, e futuros. 
Na segunda parte novos protocolos serão analisados usando o método aproximativo.
