%\usepackage{amsmath,amsthm,amssymb}
\usepackage{color}
\usepackage{nicefrac}
%\usepackage{stmaryrd} % for \llbracket and \rrbracket
%\usepackage{graphicx}
\usepackage{multirow}
\usepackage{rotate}
\usepackage{wrapfig}
%\usepackage{hyperref}
%\usepackage{mathtools}
%\usepackage[titletoc]{appendix}
\usepackage{subcaption}

%%%%%%
% Comment
%
\newif\ifcommentson\commentsonfalse
%\newif\ifcommentson\commentsontrue
\def\mywidth{.9}        %\def\mywidth{.45}
\def\mywidthRep{.8} %\def\mywidthRep{.43}
\ifcommentson
\newcommand{\commentAA}[1]{\begin{center} \parbox{\mywidth\textwidth}{\textbf{\textcolor{black}{Comment A.}} \textcolor{red}{#1 }}\end{center}}
\newcommand{\commentAB}[1]{\begin{center} \parbox{\mywidth\textwidth}{\textbf{\textcolor{black}{Comment B.}} \textcolor{red}{#1} }\end{center}}
\newcommand{\commentMA}[1]{\begin{center} \parbox{\mywidth\textwidth}{\textbf{\textcolor{black}{Comment M.}} \textcolor{red}{#1} }\end{center}}
%
\newcommand{\replyAA}[1]{\begin{center} \parbox{\mywidthRep\textwidth}{\textbf{Reply A.} \textcolor{blue}{#1} }\end{center}}
\newcommand{\replyAB}[1]{\begin{center} \parbox{\mywidthRep\textwidth}{\textbf{Reply B.} \textcolor{blue}{#1} }\end{center}}
\newcommand{\replyMA}[1]{\begin{center} \parbox{\mywidthRep\textwidth}{\textbf{Reply M.} \textcolor{blue}{#1} }\end{center}}
%
\newcommand{\commentA}[1]{\marginpar{\footnotesize \color{red} {\bf A:} \textsf{\scriptsize #1}}}
\newcommand{\commentB}[1]{\marginpar{\footnotesize \color{red} {\bf B:} \textsf{\scriptsize #1}}}
\newcommand{\commentM}[1]{\marginpar{\footnotesize \color{red} {\bf M:} \textsf{\scriptsize #1}}}
%
\newcommand{\replyA}[1]{\marginpar{\footnotesize \color{blue} {\bf A:} \textsf{\scriptsize #1}}}
\newcommand{\replyB}[1]{\marginpar{\footnotesize \color{red} {\bf B:} \textsf{\scriptsize #1}}}
\newcommand{\replyM}[1]{\marginpar{\footnotesize \color{red} {\bf M:} \textsf{\scriptsize #1}}}
%
\else
\newcommand{\commentAA}[1]{}
\newcommand{\commentAB}[1]{}
\newcommand{\commentMA}[1]{}
\newcommand{\replyAA}[1]{}
\newcommand{\replyAB}[1]{}
\newcommand{\replyMA}[1]{}
\newcommand{\commentA}[1]{}
\newcommand{\commentB}[1]{}
\newcommand{\commentM}[1]{}
\newcommand{\replyA}[1]{}
\newcommand{\replyB}[1]{}
\newcommand{\replyM}[1]{}
\fi
%%%%%%

% Macros for this paper specifically.
%Macros for changing after review
%\newcommand{\review}[1]{\textcolor{blue}{#1}} 
\newcommand{\review}[1]{#1}
\newcommand{\cutout}[1]{\textcolor{red}{#1}}
%\newcommand{\cutout}[1]{}


%\newcommand\Cite[1]{[\textbf{\color{red}{#1}}]} % Use for a citation to be filled-in later.
\newtheorem{Theorem}{Theorem}
\newtheorem{Lemma}[Theorem]{Lemma}
\newtheorem{Corollary}[Theorem]{Corollary}
\newtheorem{Proposition}[Theorem]{Proposition}
\newtheorem{Conjecture}[Theorem]{Conjecture}
\newtheorem{Definition}[Theorem]{Definition}
\newtheorem{Example}[Theorem]{Example}

\newcommand{\calw}{\mathcal{W}}
\newcommand{\calx}{\mathcal{X}}
\newcommand{\caly}{\mathcal{Y}}
\newcommand{\calz}{\mathcal{Z}}
\newcommand{\calf}{\mathcal{F}}
\newcommand{\calg}{\mathcal{G}}

\newcommand{\cupdot}{\mathbin{\mathaccent\cdot\sqcup}}
%\newcommand{\hoper}{{}_p\oplus}
\newcommand{\hchoice}[1]{\;{{}_{\mathit{#1}}{\oplus}}\;} % operator to be used among parameters: it gives nice spacing.
\newcommand{\hchoiceop}[1]{{}_{\mathit{#1}}{\oplus}} % name of the operator, to be used when no paramaters are needed.
%\newcommand{\voper}{{}_{p}\cupdot}
\newcommand{\vchoice}[1]{\;{{}_{\mathit{#1}}{\cupdot}}\;} % operator to be used among parameters: it gives nice spacing.
\newcommand{\vchoiceop}[1]{{}_{\mathit{#1}}{\cupdot}} % name of the operator, to be used when no paramaters are needed.
\newcommand{\reals}{\mathbb{R}}
\newcommand{\dist}{\mathbb{D}}
\newcommand{\imp}{\Rightarrow}
\newcommand{\conc}{\otimes}

\newcommand{\qm}[1]{``#1''}

\newcommand{\hyperc}[2]{[#1 , #2]} % Hyper [\pi > C] created by pushing a prior \pi through a channel C
\newcommand{\ggid}{g_{\mathit{id}}} % g_id gain function
\newcommand{\vgid}{V_{\gid}} % V_{g_{id}}, the g-vulnerability corresponding to g_id
\newcommand{\pchoice}[1]{\;{{}_{\mathit{#1}}\oplus}\;}%redefined by Catuscia probabilistic choice
\newcommand{\zeromat}{\hat{0}}
\newcommand{\nullchannel}{\overline{0}} % Null channel.
\renewcommand{\equiv}{\approx}

\usepackage[bookmarks=false,linktocpage=true,colorlinks=false]{hyperref} % For use of \href{}.

% vulnerabilities
\newcommand{\vg}{V_{g}} % name of g-vulnerability
\newcommand{\priorvg}[1]{\vg\left[#1\right]} % prior g-vulnerability
\newcommand{\postvg}[2]{\vg\left[#1,#2\right]} % posterior g-vulnerability
\newcommand{\vf}{\mathbb{V}} % name of generic vulnerability function
\newcommand{\priorvf}[1]{\vf\!\left[#1\right]} % prior generiic vulnerabiltiy fuction
\newcommand{\postvf}[2]{\vf\!\left[#1,#2\right]} % posterior generiic vulnerabiltiy fuction

% math
\newcommand{\eqdef}{\ensuremath{\stackrel{\mathrm{def}}{=}}}
