%Apresente os conceitos pertinentes para que o leitor entenda o problema e sua importância. Nesse momento, o aluno pode não ter domínio completo dos trabalhos relacionados. Fazer um breve resumo das soluções já existentes na literatura/mercado e como elas se comparam ao trabalho proposto. No caso de trabalhos de cunho tecnológico, listar ferramentas que resolvem o mesmo problema sendo tratado.

% No framework, modelos de seguranca sao modelados como canais de informacao teorico.
% Definir canal
% 

No campo de QIF, sistemas de segurança são modelados como canais de informação. Um canal é definido como uma função $C: \calx \times \caly \rightarrow \reals$ onde $\calx$ é o grupo das entradas, ou valores secretos, e $\caly$ é grupo dos outputs, ou observáveis:

$$ C(x,y) = p(y|x) $$

$C(x,y)$ é a probilidade do sistema produzir o observável $y$ dado o segredo $x$, $\forall x \in \calx$ e $\forall y \in \caly$.

Por exemplo, podemos modelar um sistema de login onde $\calx$ é o grupo de todas as possíveis senhas e $\caly$ um grupo de 2 elementos, se a senha é certa ou errada. Nesse sistema apenas um elemento é correto e o restante incorreto.
\begin{table}[h!]
\centering
\begin{tabular}{|c|c c|}
    \hline
    $C$   & Correto & Incorreto  \\
    \hline 
%    $"123455"$&0&1\\
    $"123456"$&1&0\\
    Restante &0&1\\
    \hline
\end{tabular}
\caption{Representação matricial de um canal}
\label{Canal}
\end{table}
A Tabela 1 mostra como o canal do login é simples e interessante para perceber a interpretação do canal. Em sistemas de segurança é comum ter varios observáveis possíveis para cada segredo(Tab. 2).
\begin{table}[h!]
\centering
\begin{tabular}{|c|c c c|}
    \hline
    $C$   & $y_1$ & $y_2$ & $y_3$  \\
    \hline 
    $x_1$&0.4&0.3&0.3\\
    $x_2$&0.2&0.8&0\\
    \hline
\end{tabular}
\caption{Representação matricial de um canal}
\label{Canal}
\end{table}

Para entender como a informação vaza é preciso modelar também o \emph{atacante}, que conhece o observável e como o sistema funciona. Alem disso, o adversário pode saber alguma coisa sobre o segredo antes do sistema executar que é defindo como a distribuição a priori $\pi$ em $\calx$ sobre os segredos.

O conhecimento a priori 
