In this work we presented a systematic way of deriving channels
representing the behavior of security protocols, and used
these channels to derive robust information-flow guarantees
about these protocols.
More precisely, we provided the first analyses of additive
and multiplicative $g$-capacities of two versions of the
Dining Cryptographers and the Crowds anonymity protocols.
The bounds provided hold irrespectively of the probability 
distribution on secret values, or of the interests and goals 
of an adversary, constituting, to the best of our knowledge,
the most general information-flow analyses of such protocols 
ever performed.

Future work could lead to a general purpose tool support to allow 
the computation of critical information flow properties.
Moreover, we want to explore algorithms for computing capacities
for systems whose possible contexts of execution are limited
in a more restricted set of priors and gain-functions.

%It also reveals results of  $\mathcal{L}^+_\forall \hyperc{\pi}{C}$ 
%for the anonymity protocols Crowds and Dining Cryptographers, which 
%has not been done before. The amount of data collected is large and 
%the possibilities of different analyses are endless, so what has
% been done so far is in no way exhaustive.
%
%The results found on this project also raise a great number of questions. 
%The better performance of the complete variation over the cycle one of the 
%Dining Cryptographers suggest that the first family of channel might be 
%always more secure than the second one, regarding the channel ordering 
%proposed in \cite{robust}. 
%The same can be said about the regular variation and the grid one 
%regarding the Crowds protocol.
%
%A natural next step, thus, would be to try to prove that this ordering hold 
%for this channels analytically, or, if it is the case, find an example of 
%$g$ and $\pi$ where that prove our conjecture wrong.
%
%There is also still a lot of analyses that can be done with the data we 
%acquired this far. 
%There is a wide range of less immediate priors and gain functions that 
%might be useful in certain contexts, therefore an analyses of the 
%capacities regarding these might be of future interest.
% Furthermore, we also wish to improve our algorithms, maybe even 
% devising some way to calculate $\mathcal{L}^+_g\hyperc{\forall}{C}$,
%  which is NP-hard efficiently enough, to enable us to derive results 
%  for small channels, at least